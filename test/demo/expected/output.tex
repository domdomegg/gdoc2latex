\title{%
  \Huge{\textbf{gdoc2latex}}
  \\
  \Large{Converts Google Docs files to LaTeX\\~\\~Your Name\\~Department of Something\\~University of Somewhere}
}
\documentclass[12pt]{article}

\usepackage[a4paper,left=2.54cm,right=2.54cm,top=2.54cm,bottom=2.54cm]{geometry}
\usepackage[T1]{fontenc}
\usepackage[utf8]{inputenc}
\usepackage{lmodern}
\usepackage{parskip}
\usepackage{hyperref}
\usepackage{graphicx}
\usepackage{adjustbox}
\usepackage{natbib}
\usepackage[dvipsnames]{xcolor}
\usepackage{minted}

\setlength\parindent{0em}

\bibliographystyle{abbrvnat}
\setcitestyle{super,citesep={,},open={},close={}}

\definecolor{lightgrey}{rgb}{0.95,0.95,0.95}
\setminted{bgcolor=lightgrey}
\setmintedinline{bgcolor=lightgrey,escapeinside=||,mathescape=true}

\renewcommand{\thefootnote}{\roman{footnote}}

\usepackage{titling}
\predate{}
\date{}
\postdate{}
\preauthor{}
\author{}
\postauthor{}

\setcounter{tocdepth}{1}

\begin{document}

\maketitle



\section{Usage}\label{id:h.9brqho78gj4b}
Here’s some content, originally written in Google Docs. It demonstrates using some of the features of gdoc2latex. It was downloaded with File > Download > Web page (.html).

{\centering \begin{figure}[h!]
  \centering
  \includegraphics[width=0.578\linewidth]{images/image1.png}
  \caption{Downloading a document from Google Docs}
\end{figure} \par}
\section{Supported features}\label{id:h.3pfyws1px6lp}
Things like \textbf{bold} and \textit{italics} among other things are supported.

Other features include:
\begin{itemize}
  \item Lists
  \item \underline{Underline}
  \item References (use BibTeX in a Google Docs footnote)\cite{ref1}
  \item Footnotes\footnote{Like this! One limitation though, they can’t start with an @ symbol}
  \item Both \textsuperscript{superscript} and \textsubscript{subscript}
  \item Links to the \underline{\href{https://github.com/domdomegg/gdoc2latex}{web}}, to an \underline{\href{mailto:someone@example.com}{email address}} or \underline{\hyperref[id:h.9brqho78gj4b]{within a document}}
  \item Working “double quotes” and ‘single quotes’
  \item Comments\hyperref[id:cmnt1]{[a]}
\end{itemize}

Plus both \mintinline{text}{|inline|} code snippets and block snippets:

\begin{minted}[breaklines]{text}
These both use the minted package, and follow roughly markdown syntax… ish?

It should also keep
the line spacing correct
\end{minted}

You can also write $math$ and use \LaTeX\ directly as an escape hatch (escaping uses the \textbackslash \ character, e.g. \$20 and \textbackslash LaTeX).




\hyperref[id:cmnt_ref1]{[a]}Like this!

\bibliography{output}

\end{document}